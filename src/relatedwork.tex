\chapter{Related Work}

\section{{\IterComp}}

\textbf{Specific work in {\IterComp} about multiple inputs and online scenarios}

\cite{chen10,chen12a} evaluate the effectiveness of iterative compilations across a large number of input test cases.
Their main motivation is to answer the question:
\textit{How data input dependent is {\itercomp}?}
Their results show that it is possible to find a combination of compiler optimisations that achieves at least 86\% of the maximum speedup across all input test cases.
These optimal combinations are program-specific and yield average speedups up to 3.75$\times$ over the highest optimisation level of compilers.

When optimising a program, the main method for {\itercomp} used by \cite{chen10,chen12a} evaluates each combination of compiler optimisations across all the available inputs, i.e., if $N$ is the number of input test cases and $M$ is the total number of combinations of compiler optimisations, they perform a total of $O(NM)$ runs of the program being optimised.
Furthermore, they use a pre-defined set of only 300 different combinations of compiler optimisations, which represents a very small sample of the optimisation search space for most modern compilers, e.g.
LLVM has 56 distinct optimisation passes and GCC has about 47 high-level (SSA form) optimisation passes and about 25 low-level (RTL) optimisation passes, which in both cases result in much more than $2^{50}$ distinct combinations of compiler optimisations, without considering repetition.

Recent work~\citep{chen12b,fang15} have applied the same idea of performing input-dependent {\itercomp} to distributed applications on data centres.
In summary, each worker receives a subset of the input dataset, called the evaluation dataset, to perform an \textit{online} {\itercomp} of the code being executed.
Each worker performs the same the method for {\itercomp} used by \cite{chen10,chen12a}, i.e., they evaluate each combination of compiler optimisations across all the evaluation dataset.
However, because the optimisation is performed online, they usually consider a small evaluation dataset and a small number of compiler optimisations.

\cite{fursin07} addressed the problem of comparing the effect of two optimisations on two distinct inputs. For that purpose, they proposed to use instruction per cycle (IPC) as the metric for performing such comparison.
Their result show that using IPC seems promising as a robust metric for {\itercomp} across large input datasets.
However, some specific optimisation techniques may affect the use of IPC as a robust metric, and specially IPC has been shown to provide poor performance estimation for multi-threaded programs~\citep{alameldeen06,eyerman08}.
In particular, IPC can give a skewed performance measure if threads spend time in \textit{spin-lock loops} or other synchronisation mechanisms. 
Some existing work on performance assessment suggest that total execution time should be used for measuring performance of multi-threaded programs~\citep{alameldeen06,eyerman08}.
\cite{alameldeen06} in particular suggest that a simple work-related metric should be used if the unit of work is representative enough.
Work-related metrics have already been largely used for measuring performance of throughput-oriented applications, for other applications, however, choosing an appropriate unit of work can be more challenging~\citep{alameldeen06}.

\section{Work and Input Size Metrics}

\textbf{Talk about work-based metrics}

Previous work have proposed profiling-based mechanism to estimate input sizes~\citep{zaparanuks12,coppa14}.
\cite{coppa14} in particular propose the concept of \textit{read memory size} for automatically estimating the size of the input passed to a routine, where \textit{read memory size} represents the number of distinct memory cells first accessed by a read operation.
In other words, the \textit{read memory size} metric measures the size of the useful portion of the input's memory footprint.
However, because we are interested in the amount of computational work performed in respect of a given input, the memory footprint of the input may not always have a direct correspondence to  the amount of computational work.

\cite{goldsmith07} use \textit{block frequency} as the measure for performance for empirically describing the asymptotic behaviour of programs, which is known as empirical computational complexity.
Block frequency is a relative metric that represents the number of times a basic block executes~\citep{ball94,ball96}.
They argue in favour of block frequency due to its portability, repeatability and exactness, since it does not suffer from timer resolution problems or non-deterministic noises.
Block frequency also has the advantage of being efficiently profiled by means of automatic code instrumentation~\citep{knuth73,ball94}.

However, in the context of comparing different optimisations, although block frequency would be able to capture aspects of optimisations that simplify the control-flow graph (CFG), measuring work at the basic block resolution would not capture effects of optimisations at the instruction level.
Because of that, we extend the idea of using basic block frequency to measure computational work by also considering the computational cost of each basic block.
The computational cost of a basic block is given by weighting the instructions that it contains.

\section{Optimal Instrumentation} \label{subsec:optimalInstrumentation}

In order to profile  block frequency, the program can be instrumented with counters that determine how many times each basic block in a program executes.
A naive instrumentation would consist basically of having a counter for each basic block which is incremented every time the basic block is reached.
Although the naive instrumentation was commonly used in practice~\citep{knuth71}, it is a very invasive instrumentation that imposes an unnecessarily high overhead in the instrumented program.
An optimal instrumentation based on the principle of \textit{conservation of flow} (\textit{Kirchhoff's first law}\footnote{Gustav Kirchhoff defined two equalities about electric circuits, known as Kirchhoff's circuit laws. The first one is about current and and the second about potential difference.}) have been originally proposed by \cite{nahapetian73} and \cite{knuth73}.
While \cite{knuth73} proposed an optimal solution for basic block profiling with \textit{vertex counters}, \cite{ball94} showed that an optimal basic block profiling with \textit{edge counters} provides the best instrumentation for block frequency profiling.
Further overhead reduction of the optimal instrumentation was later proposed by placing the counters in edges that are less likely to be executed~\cite{forman81,ball94}.

\begin{definition}[Kirchhoff's first law]
The amount of flow into a vertex equals the amount of output flow, i.e. the sum of the incoming edges of a vertex equals the sum of outgoing edges of the same vertex.
\end{definition}

The optimal instrumentation places probes in edges as the basic block frequency can be derived by summing either the flow of the incoming or outgoing edges.
However, it uses the Kirchhoff's first law in order to place probes in subset of the edges that allows to later infer the flow of all edges.
Previous work have shown that a set of edges represents the minimum number of probes for profiling block frequency if and only if the complementary set of edges forms a spanning tree~\citep{nahapetian73,ball94}.
In other words, after determining a spanning tree of the CFG, probes need to be placed only in the edges from the complement of a spanning tree, usually called \textit{cotree}.
Because the edge frequencies satisfy Kirchhoff's first law, each edge flow can be uniquely determined as an algebraic sum of the known edge flows from the cotree~\citep{nahapetian73,ball94}.

The optimal block frequency instrumentation happens in two main stages:
\textit{(i.) Before execution.} The code is instrumented with the edge counters, i.e., it requires one global counter for each edge selected to contain a probe.
\textit{(ii.) After execution.} The information from the recorded probes is propagated in the CFGs of the program, in a post-processing phase.
\lstlistingname~\ref{lst:populateEdgeFlows} shows the algorithm for the post-processing of a CFG, once we have the profiling information collected by the probes.

\begin{lstlisting}[caption={Post-processing of the CFG for populating all edge flows based on the collected probes.}, label={lst:populateEdgeFlows}, float]
// Inputs: CFG with the known edges flows from the cotree (collected probes).
// Output: Updated CFG with all edge flows.
populateEdgeFlows(G) {
  changed = true
  while changed:
    changed = false
    for B in G.vertices():
      unIN = count( G.unknownIncomingEdges(B) )
      unOUT = count( G.unknownOutgoingEdges(B) )
      if unIN==0 and unOUT==1:
        //sum known incoming and outgoing edges in B
        sIN = sum( G.incomingEdges(B) )
        sOUT = sum( G.outgoingEdges(B) )
        //update unknown outgoing edge in B with (sIN-sOUT)
        G.setUnknownOutgoingEdge(B, (sIN-sOUT))
        changed = true
      if unIN==1 and unOUT==0:
        //sum known incoming and outgoing edges in B
        sIN = sum( G.incomingEdges(B) )
        sOUT = sum( G.outgoingEdges(B) )
        //update unknown incoming edge in B with (sOUT-sIN)
        G.setUnknownIncomingEdge(B, (sOUT-sIN))
        changed = true
}
\end{lstlisting}

\lstlistingname~\ref{lst:populateEdgeFlows} is guaranteed to terminate because the probed edge flows on the complement of a spanning tree are necessary and sufficient to compute all edge flows~\citep{nahapetian73,forman81}.
Intuitively, if all the edge flows are known for the complement of a spanning tree then at any leaf of the spanning tree there is only one unknown edge flow.
This unknown edge flow can be calculated by Kirchhoff's first law.
This process repeats until all the unknown edge flows have been calculated.
Although this instrumentation algorithm is proved to produce the optimal placement of probes for well-structured CFGs, it may produce sub-optimal placement for some unstructured CFGs~\citep{ball94}.

This briefly described proof suggests that the edges can be more efficiently populated by a bottom-up propagation in the spanning tree.
By performing a post-order traversal of the spanning tree, i.e. starting from the leaves, we can then apply the flow equation from the Kirchhoff's first law.
At each node of the spanning tree, we first sum the known incoming and outgoing edges, and then the unknown edge flow will by computed by subtracting the minimum of the two sums from the maximum (as before).
This bottom-up propagation allows to populate the edge flows in a single pass over the basic blocks.

\cite{forman81} and \cite{ball94} propose to optimise the placement of the probes with respect to edges that are less likely to be executed.
It works by considering a weighting that assigns a non-negative value to each edge in the CFG.
The overhead cost of profiling a set of edges is considered to be proportional to the sum of the weights of the edges.
These weights can be obtained either by empirical measurements from previous executions or by static heuristic estimations at compile-time.
In order to minimise the profiling overhead, the instrumentation computes the maximum spanning tree in order to avoid probing in frequently executed edges.

Notice that when instrumentation is performed guided by empirical measurements from previous executions, it means that edge profiling information is used in order to produce a better edge profiling instrumentation.
The next section presents the main algorithms for producing static estimates for the weights of the edges in a CFG.

\subsection{Static Estimates of Edge Frequencies}

\cite{ball93} presented a simple algorithm that predicts the outcome of conditional branches with a reasonably good accuracy.
For this purpose, they used several branch heuristics that were derived by measuring, on a large number of programs, the probability of branches being taken in respect of some \textit{ad-hoc} features from the programs.
Their algorithm selects, for each branch, the first heuristic that applies to the branch, in a given priority order of the heuristics.
The \textit{ad-hoc} heuristics defined by \cite{ball93} are:
\begin{description}
\item[Loop branch heuristic (probability 88\%):] Probability of an edge back to a loop's head being executed.
\item[Loop exit heuristic (probability 80\%):] Probability that a comparison inside a loop will \textit{not} exit the loop. This heuristic does not apply to latch blocks, i.e. basic blocks that contain a branch back to the header of the loop.
\item[Pointer heuristic (probability 60\%):] Probability that a comparison between two pointers, where one of them can be a null pointer, will fail.
\item[Opcode heuristic (probability 84\%):] Probability that a comparison of an integer being less than zero, less than or equal to zero, or equal to a constant will fail.
\item[Guard heuristic (probability 62\%):] For a comparison with a register as operand, where the register is defined in a successor basic block which is not a post-dominator. The probability that the successor basic block is reached.
\item[Loop header heuristic (probability 75\%):] Probability of reaching a successor block that is a loop header (or pre-header) but not a post-dominator.
\item[Call heuristic (probability 78\%):] Probability of reaching a successor block that is not a post-dominator but contains a function call.
\item[Store heuristic (probability 55\%):] Probability of reaching a successor block that is not a post-dominator but contains a store instruction.
\item[Return heuristic (probability 72\%):] Probability of reaching a successor block that contains a return instruction.
\end{description}

\cite{wu94} proposed an algorithm for statically estimating edge frequencies, which improves on the work of \cite{wagner94} and \cite{ball93}.
This algorithm is able to combine several heuristics of the outcome of a branch into an estimated probability of the branch being taken.
\cite{wu94} use the \textit{ad-hoc} heuristics defined by \cite{ball93} as their initial predictions.
They also use the Dempster-Shafer theory~\citep{shafer76} that provides the necessary mathematical technique for combining evidences from the different heuristics in order to produce more accurate estimates.
These branch probabilities can then be used to estimate execution frequencies
for all the edges in a CFG.

\section{Summary}
